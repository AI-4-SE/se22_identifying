%% !TeX encoding = UTF-8
%% !TeX program = pdflatex
%% !BIB program = bibtex
%%
%%% Um einen Artikel auf deutsch zu schreiben, genügt es die Klasse ohne
%%% Parameter zu laden.
\documentclass[]{lni}
%%% To write an article in English, please use the option ``english'' in order
%%% to get the correct hyphenation patterns and terms.
%%% \documentclass[english]{class}
%%
\begin{document}
%%% Mehrere Autoren werden durch \and voneinander getrennt.
%%% Die Fußnote enthält die Adresse sowie eine E-Mail-Adresse.
%%% Das optionale Argument (sofern angegeben) wird für die Kopfzeile verwendet.
\author[Vorname1 Name1\and Vorname2 Name2]
{Vorname1 Nachname1\footnote{Einrichtung/Universität, Abteilung, Anschrift, Postleitzahl Ort, \email{emailadresse@author1}}\and
 Vorname2 Nachname2\footnote{Einrichtung/Universität, Abteilung, Anschrift, Postleitzahl Ort, \email{emailadresse@author2}}
 und weitere Autorinnen und Autoren in der gleichen Notation}
\title[Kurztitel (falls nötig)]{Titel}
\startpage{11}% Beginn der Seitenzählung für diesen Beitrag
\editor{Herausgeber et al.}
\booktitle{Name-der-Konferenz}
\year{2017}
\maketitle

\begin{abstract}
Hier kommt die Zusammenfassung hin
Add your abstract here
\end{abstract}
\begin{keywords}
Hier kommen die Keywords hin
Add your keywords here
\end{keywords}
%%% Beginn des Artikeltexts
\section{Überschrift}

%%% Angabe der .bib-Datei (ohne Endung)
\bibliography{mybibfile}
\end{document}
