% !TeX encoding = UTF-8
% !TeX program = pdflatex
% !BIB program = bibtex
% !TeX spellcheck = de_DE

%%% Um einen Artikel auf deutsch zu schreiben, genügt es die Klasse ohne
%%% Parameter zu laden.
\documentclass[]{lni}
%%% To write an article in English, please use the option ``english'' in order
%%% to get the correct hyphenation patterns and terms.
%%% \documentclass[english]{class}
%%
\begin{document}
%%% Mehrere Autoren werden durch \and voneinander getrennt.
%%% Die Fußnote enthält die Adresse sowie eine E-Mail-Adresse.
%%% Das optionale Argument (sofern angegeben) wird für die Kopfzeile verwendet.
\title[Ein Kurztitel]{Ein sehr langer Titel über mehrere Zeilen mit sehr vielen
Worten und noch mehr Buchstaben}
%%%\subtitle{Untertitel / Subtitle} % if needed
\author[Vorname1 Nachname1 \and Firstname2 Lastname2]
{Vorname1 Nachname1\footnote{Universität, Abteilung, Straße, Postleitzahl Ort,
Land \email{emailaddress@author1}} \and
Firstname2 Lastname2\footnote{University, Department, Address, Country
\email{emailaddress@author2}}}
\startpage{11} % Beginn der Seitenzählung für diesen Beitrag / Start page
\editor{Herausgeber et al.} % Names of Editors
\booktitle{Name-der-Konferenz} % Name of book title
\year{2017}
%%%\lnidoi{18.18420/provided-by-editor-02} % if known
\maketitle

\begin{abstract}
This is a brief overview of the paper, which should be 70 to 150 words long and
include the most relevant points. This has to be a single paragraph.
\end{abstract}
\begin{keywords}
Schlagwort1 \and Schlagwort2 %Keyword1 \and Keyword2
\end{keywords}
%%% Beginn des Artikeltexts
\section{Überschrift/Heading}

%%% Angabe der .bib-Datei (ohne Endung) / State .bib file (for BibTeX usage)
\bibliography{mybibfile} %\printbibliography if you use biblatex/Biber
\end{document}
