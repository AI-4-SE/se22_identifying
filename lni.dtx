% \iffalse meta-comment
%<*internal>
\iffalse
%</internal>
%<*readme>
# lni

```
----------------------------------------------------------------------------
lni      -- A class for submissions to the ``Lecture Notes in Informatics''
(c) 2016 
Version:    1.0
Maintainer: Martin Sievers
Email:      martin.sievers@schoenerpublizieren.de
License:    Released under the LaTeX Project Public License v1.3c or later
See:        http://www.latex-project.org/lppl.txt
----------------------------------------------------------------------------
```

This is the current version of the class “lni” for submissions to the 
“[Lecture Notes in Informatics]” published by the “Gesellschaft für Informatik”
([GI]).

It is based on previous templates created on behalf of the GI.

Stable versions are always uploaded to CTAN. In addition you'll find the most 
recent developer version on GitHub at <https://github.com/sieversMartin/lni>.

Please note, that this is not yet an official release!

## Changelog

* v1.0: First upload of the reworked version

  [GI]: https://www.gi.de/
  [Lecture Notes in Informatics]: https://www.gi.de/service/publikationen/lni.html
%</readme>
%<*internal>
\fi
\def\nameofplainTeX{plain}
\ifx\fmtname\nameofplainTeX\else
  \expandafter\begingroup
\fi
%</internal>
%<*install>
\input docstrip.tex
\keepsilent
\askforoverwritefalse
\declarepreamble\bibtexengpre
%% 
%% lnig.bst
%% Lecture Notes in Informatics Style File (english)
%% Version 1.0 (2014/12/17)
%%
%%
%% The Style File is based on alpha.bst
%%
%% This work may be freely used,
%% distributed and modified.
%% Modified files should be clearly indicated as such and renamed.
%%
%%***************************************************      
% Changelog:
%
%  1.0       (17.12.14), Judith Michael, Alpen-Adria-Universität Klagenfurt, Austria
% Changes:  
%  - Last Name before the prename
%  - ; between Autors
%  - : before title
%  - Lable: Abbreviation of name of first author instead of + (former: A+09 or WFEL14, now Az09 and Wa14)
%  - Lable: Use first 2 letters of title if there is no author
%  - add : after In (functions inproceedings and inbook) 
%  - no emphasize of titles
%  - editors in bracelets ()
%  - correct order of reference infos
%  - correct signs (.,:()) between reference infos 
%  - Add second prename
%  - change order of series (not: "number" in "series", now "series" "number"
%
%%***************************************************
%
\endpreamble

\declarepreamble\bibtexgerpre
%% 
%% lnig.bst
%% Lecture Notes in Informatics Style File (german)
%% Version 2.0 (2014/12/17)
%%
%% The Style File is based on alpha.bst
%%
%% This work may be freely used,
%% distributed and modified.
%% Modified files should be clearly indicated as such and renamed.
%%
%%***************************************************      
% Changelog:
%
%  1.0      (21.09.90),  Robert Tolksdorf, K\"ortestr. 30, D-1000,  Berlin 61, West Germany
% Adapted bibliography style `geralpha' that changes several english-language
% ``constants'' in the style.
%
%  2.0       (17.12.14), Judith Michael, Alpen-Adria-Universität Klagenfurt, Austria
% Changes:  
%  - Last Name before the prename
%  - ; between Autors
%  - : before title
%  - Lable: Abbreviation of name of first author instead of + (former: A+09 or WFEL14, now Az09 and Wa14)
%  - Lable: Use first 2 letters of title if there is no author
%  - add : after In (functions inproceedings and inbook) 
%  - no emphasize of titles
%  - editors in bracelets ()
%  - correct order of reference infos
%  - correct signs (.,:()) between reference infos 
%  - Add second prename
%  - change order of series (not: "number" in "series", now "series" "number"
%
%%***************************************************
%
\endpreamble

\declarepreamble\classpre
-------:| -----------------------------------------------------------------
    lni:| A class for submissions to the ``Lecture Notes in Informatics''
 Author:| Martin Sievers
  Email:| martin.sievers@schoenerpublizieren.de
License:| Released under the LaTeX Project Public License v1.3c or later
    See:| http://www.latex-project.org/lppl.txt

\endpreamble
\postamble

Copyright (C) 2016 by Martin Sievers <martin.sievers@schoenerpublizieren.de>

This work may be distributed and/or modified under the
conditions of the LaTeX Project Public License (LPPL), either
version 1.3c of this license or (at your option) any later
version.  The latest version of this license is in the file:

http://www.latex-project.org/lppl.txt

This work is "maintained" (as per LPPL maintenance status) by
Martin Sievers.

This work consists of the file  lni.dtx
                                lni.ins
                                README.md
          and the derived files lni.pdf
                                lni.cls
                                lni-author-template.tex
                                lni.bst
                                lnig.bst
\endpostamble

\usedir{tex/latex/lni}
\AddGenerationDate
\generate{
  \usepreamble\classpre
  \file{\jobname.cls}{\from{\jobname.dtx}{class}}
}
\usedir{bibtex/bst/lni}
\generate{
  \usepreamble\bibtexengpre
  \file{\jobname.bst}{\from{\jobname.dtx}{bibtex,eng}}
  \usepreamble\bibtexgerpre
  \file{\jobname g.bst}{\from{\jobname.dtx}{bibtex,ger}}
}
%</install>
%<install>\endbatchfile
%<*internal>
\usedir{source/latex/lni}
\generate{
  \usepreamble\classpre
  \file{\jobname.ins}{\from{\jobname.dtx}{install}}
}
\nopreamble\nopostamble
\usedir{doc/latex/lni}
\generate{
  \file{README.md}{\from{\jobname.dtx}{readme}}
  \file{lni-author-template.tex}{\from{\jobname.dtx}{template}}
}
\ifx\fmtname\nameofplainTeX
  \expandafter\endbatchfile
\else
  \expandafter\endgroup
\fi
%</internal>
% \fi
%
% \iffalse
%<*driver>
\ProvidesFile{lni.dtx}
%</driver>
%<class>\NeedsTeXFormat{LaTeX2e}[1999/12/01]
%<class>\ProvidesClass{lni}
%<*class>
    [2016/07/04 v1.0 A class for submissions to the ``Lecture Notes in Informatics'']
%</class>
%<*driver>
\documentclass[a4paper]{ltxdoc}
\usepackage[english,ngerman]{babel}
\usepackage[utf8]{inputenc}
\usepackage[T1]{fontenc}
\usepackage{libertine}
\usepackage[scaled=0.85]{beramono}
\usepackage[final,tracking=smallcaps,expansion=alltext,protrusion=true]{microtype}%
\SetTracking{encoding=*,shape=sc}{50}%
\usepackage{textcomp}
\usepackage{upquote}
%\usepackage{array,booktabs,tabularx}
\usepackage[final]{listings}
%\usepackage[onehalfspacing]{setspace}
\usepackage{xspace}
\usepackage[dvipsnames]{xcolor}
\newcommand*{\MiKTeX}{\textsf{MiK\TeX{}}}
\usepackage[%
	pdftitle={lni: A LaTeX class for submissions to the ``Lecture Notes in 
   	Informatics'', published by the GI},
   pdfauthor={Martin Sievers},
   urlcolor=blue,%
	linktocpage,%
	colorlinks=true]{hyperref}
	
\newcommand{\lni}{\texttt{lni}}	
\newcommand{\LNI}{Lecture Notes in Informatics}	
\makeatletter
\newcommand*\DescribeOption{%
\leavevmode%
\@bsphack%
\begingroup%
 \MakePrivateLetters%
 \Describe@Option%
}%
\newcommand*\Describe@Option[1]{%
 \endgroup%
\marginpar{%
 \raggedleft%
 \PrintDescribeEnv{#1}%
}%
\SpecialOptionIndex{#1}%
\@esphack%
\ignorespaces%
}%
\newcommand*\SpecialOptionIndex[1]{%
\@bsphack%
\index{%
 #1\actualchar{\protect\ttfamily#1} (option)\encapchar usage%
}%
\index{%
 options:\levelchar#1\actualchar{\protect\ttfamily#1}%
 \encapchar usage%
}%
\@esphack%
}%
\makeatother
\lstset{
 basicstyle   = \small\ttfamily,
 gobble       = 2,
 keywordstyle = \color{blue}\bfseries,
 language     = [LaTeX]{TeX},
 moretexcs    = {,
   includegraphics,
 }
 frame        = single,
 backgroundcolor = \color{yellow!60},
 framesep     = 5pt,
 literate={Ö}{{\"O}}1 {Ä}{{\"A}}1 {Ü}{{\"U}}1 {ß}{{\ss}}1 {ü}{{\"u}}1 {ä}{{\"a}}1 {ö}{{\"o}}1 
}%
\lstnewenvironment{examplecode}[1][]
{\lstset{#1}}
{}
\providecommand*\env[1]{\texttt{#1}}
\providecommand*\file[1]{\texttt{#1}}
\providecommand*\opt[1]{\texttt{#1}}
\providecommand*\pkg[1]{\textsf{#1}}
\OnlyDescription     %nur Anleitung (ohne Index und History)
\CodelineIndex       %kein Index wenn auskommentiert
\EnableCrossrefs     %kein Index wenn auskommentiert
\RecordChanges       %keine History wenn auskommentiert
\begin{document}
  \DocInput{\jobname.dtx}
\end{document}
%</driver>
% \fi
% \CheckSum{0}
% \CharacterTable
%  {Upper-case    \A\B\C\D\E\F\G\H\I\J\K\L\M\N\O\P\Q\R\S\T\U\V\W\X\Y\Z
%   Lower-case    \a\b\c\d\e\f\g\h\i\j\k\l\m\n\o\p\q\r\s\t\u\v\w\x\y\z
%   Digits        \0\1\2\3\4\5\6\7\8\9
%   Exclamation   \!     Double quote  \"     Hash (number) \#
%   Dollar        \$     Percent       \%     Ampersand     \&
%   Acute accent  \'     Left paren    \(     Right paren   \)
%   Asterisk      \*     Plus          \+     Comma         \,
%   Minus         \-     Point         \.     Solidus       \/
%   Colon         \:     Semicolon     \;     Less than     \<
%   Equals        \=     Greater than  \>     Question mark \?
%   Commercial at \@     Left bracket  \[     Backslash     \\
%   Right bracket \]     Circumflex    \^     Underscore    \_
%   Grave accent  \`     Left brace    \{     Vertical bar  \|
%   Right brace   \}     Tilde         \~}
%
% \changes{v1.0}{2016/07/04}{Transfered old files into one dtx file}
% \changes{v1.0}{2016/07/04}{Added \pkg{mmap}, changed options for \pkg{microtype} and added macro \cs{powerset}}
%
% \GetFileInfo{\jobname.dtx}
% \DoNotIndex{\newcommand,\newenvironment}
%
%\title{\textsf{lni} --- A class for submissions to the ``Lecture Notes in Informatics''\thanks{This file
%   describes version \fileversion, last revised \filedate.}
%}
%\author{Martin Sievers\thanks{Email: martin.sievers@schoenerpublizieren.de}}
%\date{Released \filedate}
%
%\maketitle
%
%\changes{v1.0}{2016/05/11}{First public release}
%
% \begin{abstract}
% 
% \end{abstract}
%
% \section{Introduction}
%
% \section{Installation}
% The \lni{} bundle is distributed via \href{https://github.com/sieversMartin/LNI}{Github} and 
% \href{www.ctan.org}{CTAN}. The later is the basis for all updates of the two 
% main \TeX{} distributions \MiKTeX{} and 
% \TeX{}~Live. Thus the easiest way to get all files needed to typeset an 
% article for the \LNI{} is to use the package manager for your distribution.
% 
% 
% \section{Usage}
% To use the predefined layout for the \LNI{} just load the class file as usual 
% with \cs{documentclass\{lni\}}.
%
% The class file loads a bunch of packages which are all part of modern \TeX{} 
% distributions. Therefore, if you are confronted 
% with a missing package, please try to download and install them using your 
% distribution's package manager. Alternatively got to \href{www.ctan.org}{CTAN} 
% to download missing packages.
%
% \subsection{Options}
% Although the class file includes all layout information for a submission to 
% the \LNI{}, there are options to adapt the output one way or another.
%
% \DescribeMacro{english}A document loading the \lni{} class file, uses German 
% language adoptions by default. To switch to English, just load the class with 
% option \opt{english}.
%
% The language influences not only the hyphenation patterns and terms used in 
% the text, but also the choice of a corresponding \BibTeX{} file 
% (cf.~\ref{bibliography}).
%
% \DescribeMacro{utf8}\DescribeMacro{latin1}\DescribeMacro{applemac}Although 
% nowadays all major plattform support and widely use UTF-8 encoding for 
% working with text files, there might be some need to change the input 
% encoding the \LaTeX{} document uses.
%
% This can be achieved by giving one of the options \opt{utf8} (which is the 
% default), \opt{latin1} or \opt{applemac} to the document class.
%
% \section{Setting up a document}
% You can use the file \file{lni-author-template.tex} as a good starting point 
% for setting up a document for submission. The \lni{} class uses the standard 
% ways to build an article.
% \subsection{Special macros}
% \cs{startpage}
%
% \subsection{Title page}
% \DescribeMacro{\title}The title of your work is given using the \cs{title} 
% macro. In addition to the title itself, you can add a short title to be used 
% in the header of a page:\\
% \cs{title\oarg{Short title}\marg{Title}}.
%
% \DescribeMacro{\author}\DescribeMacro{\email}The authors of an article are 
% given using the standard \cs{author} macro. Multiple authors are separated 
% by \cs{and}; the affiliations have to be added with 
% \cs{footnote\marg{affiliations}} where you can use 
% \cs{email\marg{email address}} for the email address of an author.
%
% Finally \cs{maketitle} will output the formatted title page.
%
% \subsection{Main text}
% \subsubsection{Headings}
% \DescribeMacro{\section}\DescribeMacro{\subsection}\DescribeMacro{\subsubsection}
% You can use the standard macros \cs{section}, \cs{subsection}, \dots{} for 
% sectioning your text.
% 
% \subsubsection{Footnotes}
% \DescribeMacro{\footnote}%
% For adding a footnote, just use \cs{footnote\marg{footnote text}} where 
% needed. Please note, that the footnote counter is automatically set to the 
% correct value at the beginning of your text, i.\,e. it respects the number 
% of affiliations given on the title page.
%
% \subsubsection{Lists}
% \DescribeEnv{itemize}\DescribeEnv{enumerate}%
% The \lni{} class redefines the standard lists environments \env{itemize} and 
% \env{enumerate} to meet the requirements of the \LNI{}.
%
% \subsubsection{Floating objects}
% \DescribeEnv{figure}\DescribeEnv{table}%
% The environments \env{figure} and \env{table} can be used the standard way to 
% include graphics or tables.
%
% However, please note, that the default placement parameters are changed to 
% ``htbp'' by the class \lni{}. If you need some local adjustment, please use 
% the optional argument of both environments (cf.~Listing~\ref{lst:figure}).
%
% \DescribeMacro{\caption}\DescribeMacro{\label}
% A caption should be added by \cs{caption\marg{caption text}}, followed 
% immediately by a \cs{label\marg{Unique label}} entry.
% \begin{examplecode}[label={lst:figure}]
% \begin{figure}[tb]
% \includegraphics{...}
% \caption{...}
% \label{...}
% \end{figure}
% \end{examplecode}
%
% If you want to center floats, please \emph{do not} use the \env{center} 
% environment, but the macro \cs{centering}, which does not add extra white 
% space (cf.~Listing~\ref{lst:table}).
% \begin{examplecode}[label={lst:table}]
% \begin{table}
% \centering
% \begin{tabular}{lll}
% ...
% \end{tabular}
% \caption{...}
% \label{...}
% \end{table}
% \end{examplecode}
%
% \subsubsection{Listings/Source code}
% Verbatim Listings
% \subsubsection{Equations}
%
% \subsection{Bibliography}
% 
% \section{}
%
% \section{Bugs and feature request}
% 
%\StopEventually{^^A
%  \PrintChanges
%  \PrintIndex
%}
%
% \section{Implementation}
%
%    \begin{macrocode}
%<*class>
%    \end{macrocode}
% Satzspiegelposition gemäß LNI-Herausgeberrichtlinien (Korrektur von 
%Thomas.Kuehne@mcs.vuw.ac.nz)
%    \begin{macrocode}
\setlength{\hoffset}{-0,2cm}% correct left margin to 4.2cm
\setlength{\voffset}{2.60cm}% correct top margin to 5.25cm
%
\def\@clearglobaloption#1{%
  \def\@tempa{#1}%
  \def\@tempb{\@gobble}%
  \@for\next:=\@classoptionslist\do
    {\ifx\next\@tempa
       \message{Cleared  option \next\space from global list}%
     \else
       \edef\@tempb{\@tempb,\next}%
     \fi}%
  \let\@classoptionslist\@tempb
  \expandafter\ifx\@tempb\@gobble
    \let\@classoptionslist\@empty
  \fi}

\DeclareOption{latin1}{\PassOptionsToPackage{latin1}{inputenc}}
\DeclareOption{utf8}{\PassOptionsToPackage{utf8}{inputenc}}
\DeclareOption{ansinew}{\PassOptionsToPackage{ansinew}{inputenc}}
\newif\iflnienglish
\lnienglishfalse
\DeclareOption{english}{\lnienglishtrue\@clearglobaloption{english}}
\newif\ifusehyperref
\usehyperreftrue
\DeclareOption{nohyperref}{\usehyperreffalse}
\ExecuteOptions{utf8}
\DeclareOption*{\ClassWarning{Option `\CurrentOption` is not supported!}}
\ProcessOptions\relax

\RequirePackage[ngerman=ngerman-x-latest]{hyphsubst}
\LoadClass[10pt,twoside,a4paper,fleqn]{article}
\RequirePackage{cmap}
\RequirePackage{inputenc}
\RequirePackage[T1]{fontenc}

\iflnienglish
   \RequirePackage[ngerman,english]{babel}
\else
   \RequirePackage[english,ngerman]{babel}
\fi

% Define a modern variant of Times as the main font
\RequirePackage{newtxtext}
\RequirePackage{newtxmath}
\RequirePackage[zerostyle=b,straightquotes,scaled=.9]{newtxtt}
\RequirePackage[final,tracking=smallcaps,expansion=alltext,protrusion=true]{microtype}%
\SetTracking{encoding=*,shape=sc}{50}%
\UseMicrotypeSet[protrusion]{basicmath} % disable protrusion for tt fonts
%    \end{macrocode}
% Introduce \cs{powerset} - hint by \url{http://matheplanet.com/matheplanet/nuke/html/viewtopic.php?topic=136492&post_id=997377}
%    \begin{macrocode}
\DeclareFontFamily{U}{MnSymbolC}{}
\DeclareSymbolFont{MnSyC}{U}{MnSymbolC}{m}{n}
\DeclareFontShape{U}{MnSymbolC}{m}{n}{
    <-6>  MnSymbolC5
   <6-7>  MnSymbolC6
   <7-8>  MnSymbolC7
   <8-9>  MnSymbolC8
   <9-10> MnSymbolC9
  <10-12> MnSymbolC10
  <12->   MnSymbolC12%
}{}
\DeclareMathSymbol{\powerset}{\mathord}{MnSyC}{180}
%    \end{macrocode}
% Satzspiegel
%    \begin{macrocode}
\setlength{\textheight}{19.2cm}
\setlength{\headheight}{20.39pt}
\setlength{\textwidth}{12.6cm}
\setlength{\topmargin}{0.17cm}%
\setlength{\oddsidemargin}{1.85cm}% korrigierte twoside werte
\setlength{\evensidemargin}{1.85cm}% korrigierte twoside werte
\setlength{\headsep}{0.31cm}%

% Ragged bottom -- verhindert die Ausdehnung der Seite = Veränderung der Abstände
\def\thisbottomragged{\def\@textbottom{\vskip\z@ plus.0001fil
\global\let\@textbottom\relax}}

% Seitenzahlen -- Größe der Box
\renewcommand\@pnumwidth{3em}
\renewcommand\@tocrmarg{3.5em}
\def\@dottedtocline#1#2#3#4#5{%
  \ifnum #1>\c@tocdepth \else
    \vskip \z@ \@plus.2\p@
    {\leftskip #2\relax \rightskip \@tocrmarg \advance\rightskip by 0pt plus 2cm
               \parfillskip -\rightskip \pretolerance=10000
     \parindent #2\relax\@afterindenttrue
     \interlinepenalty\@M
     \leavevmode
     \@tempdima #3\relax
     \advance\leftskip \@tempdima \null\nobreak\hskip -\leftskip
     {#4}\nobreak
     \leaders\hbox{$\m@th
        \mkern \@dotsep mu\hbox{.}\mkern \@dotsep
        mu$}\hfill
     \nobreak
     \hb@xt@\@pnumwidth{\hfil\normalfont \normalcolor #5}%
     \par}%
  \fi}
  
\renewcommand{\title}{\@dblarg\@@title}
\def\@@title[#1]#2{\gdef\@shorttitle{#1}\gdef\@title{#2}}

\renewcommand{\author}{\@dblarg\@@author}
\def\@@author[#1]#2{\gdef\@shortauthor{{\let\footnote\@gobble\def\and{\unskip\ \andname\ }#1}}\gdef\@author{#2}}

\newcommand*{\email}[1]{{\urlstyle{same}\protect\url{#1}}}

% Title: Kopie aus article.cls mit anderem \thispagestyle
\renewcommand\maketitle{\par%
% Keine Vertikale Einrückung über dem Titel auf der Titelseite
%\changepage{0mm}{0mm}{0mm}{0mm}{0mm}{0.56cm}{0mm}{0mm}{0mm}  
\begingroup
    \renewcommand\thefootnote{\@arabic\c@footnote}%
    \def\@makefnmark% keine Einrückung der Fußnoten eingestellt
		 {\@textsuperscript{\normalfont\@thefnmark}}%
		 \long\def\@makefntext##1{%
		 \@setpar{\@@par
		 \@tempdima = \hsize
		 \advance\@tempdima -1em
		 \parshape \@ne 0.15cm \@tempdima}%
		 \par\parindent 0.3cm \noindent
		 \hb@xt@\z@{\hss\@textsuperscript{\normalfont\@thefnmark}\,}##1}
    \if@twocolumn
      \ifnum \col@number=\@ne
        \@maketitle
      \else
        \twocolumn[\@maketitle]%
      \fi
    \else
      \newpage
      \global\@topnum\z@   % Prevents figures from going at top of page.
      \@maketitle
    \fi
    \thispagestyle{plain}
    \@thanks   
  \endgroup
  \global\let\thanks\relax
  \global\let\maketitle\relax
  \global\let\@maketitle\relax
  \global\let\@thanks\@empty
  \global\let\@author\@empty
  \global\let\@date\@empty
  \global\let\@title\@empty
  \global\let\title\relax
  \global\let\author\relax
  \global\let\date\relax
  \global\let\and\relax
}

\def\@maketitle{%
  \newpage
  \null
  \begin{center}%
  \vskip -24pt% Abstand vor dem Titel
  \raggedright % Linksbündig
  \let\footnote\thanks
    {\fontsize{14}{17}\fontseries{b}\selectfont \@title \par}%
    \vskip 19pt% Abstand nach dem Titel
    {\fontsize{10}{12}\selectfont
      \lineskip .5em%
        \@author
      \par}%
    \vskip 28pt% Abstand vor dem Abstract
  \end{center}%
  \par
}

% Abstract
\renewenvironment*{abstract}{%
 \renewcommand{\abstractname}{Abstract}
 \fontsize{9}{10}\selectfont
    \noindent{\bfseries \abstractname:}%
}{\endquotation}  

% Keywords
\newenvironment{keywords}{\fontsize{9}{10}\selectfont
    \noindent{\bfseries Keywords:}}{}

% Überschriften
\renewcommand{\section}{\@startsection{section}{1}{\z@}%
  {-16\p@ \@plus -4\p@ \@minus -4\p@}{5\p@ \@plus 4\p@ \@minus 4\p@}{\fontsize{12}{14}\fontseries{b}\selectfont}}
%  {19pt}{8pt}{\fontsize{12}{14}\fontseries{b}\selectfont}}
\renewcommand{\subsection}{\@startsection{subsection}{2}{\z@}%
  {-16\p@ \@plus -4\p@ \@minus -4\p@}{8\p@ \@plus 4\p@ \@minus 4\p@}{\fontsize{10}{12}\fontseries{b}\selectfont}}
%  {16pt}{4pt}{\fontsize{10}{12}\fontseries{b}\selectfont}}
% weitere nicht durch Formatvorlage geklärt

% Bildunterschriften
\setlength{\abovecaptionskip}{4pt}%Abstand vor der Bildunterschrift
\setlength{\belowcaptionskip}{0pt}%Abstand nach der Bildunterschrift
\setlength{\intextsep}{6pt}%Abstand nach der Grafik
\long\def\@makecaption#1#2{%
  \vskip\abovecaptionskip
  {%\fontsize{5}{6}\selectfont
   \sbox\@tempboxa{\small#1: #2}%
  \ifdim \wd\@tempboxa >\hsize
    #1: #2\par
  \else
    \global \@minipagefalse
    \hb@xt@\hsize{\hfil\box\@tempboxa\hfil}%
  \fi}
  \vskip\belowcaptionskip}

% Abbildungen
\def\fps@figure{htbp}
\def\fnum@figure{\figurename~\thefigure}
\def\@floatboxreset{%
        \reset@font
        \small
        \@setnobreak
        \@setminipage
}
\setcounter{topnumber}{10}% maximale Anzahl gleitender Objekte am Seitenanfang
\setcounter{bottomnumber}{10}% maximale Anzahl gleitender Objekte am Seitenende
\renewcommand{\topfraction}{1.0}% Anteil den gleitende Objekte am Seitenanfang einnehmen dürfen
\renewcommand{\bottomfraction}{1.0}% Anteil den gleitende Objekte am Seitenende einnehmen dürfen

% Tabellen
\def\fps@table{htbp}
\def\fnum@table{\tablename~\thetable}
\renewcommand{\arraystretch}{1.1}%Abstand zu Linien in Tabelle
\renewenvironment{table}
               {\setlength\abovecaptionskip{6\p@}%
                \setlength\belowcaptionskip{0\p@}%
                \@float{table}}
               {\end@float}
\renewenvironment{table*}
               {\setlength\abovecaptionskip{6\p@}%
                \setlength\belowcaptionskip{0\p@}%
                \@dblfloat{table}}
               {\end@dblfloat}

% Gleichungen mit richtiger Einrückung, 0.5cm
% fleqn-Option oben
\setlength{\mathindent}{0.5cm}

% listings mit richtiger einrückung, 0.5 cm
\RequirePackage{verbatim}
\def\verbatim@processline{\hskip0.5cm\the\verbatim@line\par}

% Fussnoten mit 5cm Linie und richtiger Größe
\renewcommand\footnoterule{%
  \kern-3\p@
  \hrule\@width 5cm
  \kern2.6\p@}
  \newdimen\fnindent
  \fnindent0.18cm
\long\def\@makefntext#1{%
    \parindent \fnindent%
    \leftskip \fnindent% Einrückung vor der footnotemark
    \noindent
    \llap{\hb@xt@0.10cm{\hss\@makefnmark\ }}\ignorespaces#1}


% Absatzeinzüge und Abstände
\setlength{\parindent}{0pt}
\setlength{\parskip}{8pt}

% Aufzählungen
% mit Punkt
\renewcommand*\itemize{%
  \ifnum \@itemdepth >\thr@@\@toodeep\else
  	\setlength{\labelsep}{0.70cm}%Abstand zum Aufzählungspunkt
    \advance\@itemdepth\@ne
    \edef\@itemitem{labelitem\romannumeral\the\@itemdepth}%   
    \expandafter
    \list  
      \csname\@itemitem\endcsname
      {\def\makelabel##1{\hss\llap{##1}}%
       \setlength{\itemsep}{8pt}%
       \setlength{\parsep}{-2pt}}%
       %\setlength{\itemindent}{0.5cm}
  \fi}
  
  % Nummierierte Aufzählung
  \renewcommand{\labelenumii}{\alph{enumii})}
  \renewcommand*\enumerate{%    
  \ifnum \@enumdepth >\thr@@
      \@toodeep
    \else
		\setlength{\labelsep}{0.70cm}%Abstand zur Aufzählungsnummer      
      \advance\@enumdepth \@ne
      \edef\@enumctr{enum\romannumeral\the\@enumdepth}%
    \fi
    \@ifnextchar[{\@enumlabel@{\@enumerate@}[}{\@enumerate@}}
  \def\@enumerate@{%
    \expandafter\list\csname label\@enumctr\endcsname{%
      \usecounter{\@enumctr}%
      \def\makelabel##1{\hss\llap{##1}}
		\setlength{\labelsep}{0.6cm} %Einrückung des Aufzählungszeichens     
      \setlength{\itemsep}{8pt}%
      \setlength{\parsep}{-2pt}}
  }

% bibliography
\renewenvironment{thebibliography}[1]
     {\iflnienglish\selectlanguage{english}\else\selectlanguage{ngerman}\fi
      \inputencoding{latin1}
      \section*{\refname}%
      \bgroup\fontsize{9}{10}\selectfont
      \list{\@biblabel{\@arabic\c@enumiv}}%
           {\settowidth\labelwidth{\@biblabel{#1}}%
            \leftmargin\labelwidth
            \advance\leftmargin\labelsep
            \@openbib@code
            \usecounter{enumiv}%
            \let\p@enumiv\@empty
            \renewcommand\theenumiv{\@arabic\c@enumiv}}%
      \sloppy
      \clubpenalty4000
      \@clubpenalty \clubpenalty
      \widowpenalty4000%
      \sfcode`\.\@m}
     {\def\@noitemerr
       {\@latex@warning{Empty `thebibliography' environment}}%
      \endlist\egroup\inputencoding{utf8}}

\RequirePackage{graphicx}
\RequirePackage{fancyhdr}
\RequirePackage{listings}
%\usepackage{changepage} %for changing topmargin on first page
%\usepackage[figurename=Abb., tablename=Tab., small]{caption}[2008/04/01]
\newcommand{\andname}{}
\addto\captionsngerman{%
  \renewcommand{\andname}{und}%
  \renewcommand{\figurename}{Abb.}%
  \renewcommand{\tablename}{Tab.}%
  \renewcommand{\lstlistingname}{List.}
  \renewcommand{\refname}{Literaturverzeichnis}
}

\addto\captionsenglish{%
  \renewcommand{\andname}{and}%
  \renewcommand{\figurename}{Fig.}%
  \renewcommand{\tablename}{Tab.}%
  \renewcommand{\lstlistingname}{List.}
}

\newcommand*{\startpage}[1]{\setcounter{page}{#1}}


\newcommand*{\@booktitle}{}
\newcommand*{\booktitle}[1]{\renewcommand{\@booktitle}{#1}}

\newcommand*{\@editor}{}
\newcommand*{\editor}[1]{\renewcommand{\@editor}{#1}}

\AtBeginDocument{%
   \let\@rigyear\year
   \newcommand*{\@year}{\the\@rigyear}
   \renewcommand*{\year}[1]{%
      \renewcommand{\@year}{#1}%
   }%
}%
%    \end{macrocode}
% Kopfzeileneinstellungen
%    \begin{macrocode}
\fancypagestyle{plain}{%
   \fancyhead{} % Löscht alle Kopfzeileneinstellungen
   \fancyhead[RO]{\small\@editor~(Hrsg.):\ \@booktitle,\linebreak Lecture Notes in Informatics (LNI), Gesellschaft für Informatik, Bonn~\@year \hspace{5pt}\thepage\hspace{0.05cm}}
   \fancyfoot{} % Löscht alle Fußzeileneinstellungen
   \renewcommand{\headrulewidth}{0.4pt} %Linie unter Kopfzeile 
}

\pagestyle{fancy}
\fancyhead{} % Löscht alle Kopfzeileneinstellungen
\fancyhead[RO]{\small\@shorttitle\hspace{5pt}\thepage\hspace{0.05cm}}
\fancyhead[LE]{\hspace{0.05cm}\small\thepage\hspace{5pt}\@shortauthor}
\fancyfoot{} % Löscht alle Fußzeileneinstellungen
\renewcommand{\headrulewidth}{0.4pt} %Linie unter Kopfzeile 

\RequirePackage{url}
%    \end{macrocode}
% improve wrapping of URLs - hint by http://tex.stackexchange.com/a/10419/9075
%    \begin{macrocode}
\g@addto@macro{\UrlBreaks}{\UrlOrds}

\ifusehyperref
\RequirePackage[pdfusetitle]{hyperref}
\hypersetup{%
   bookmarks=false,%
   %%%breaklinks=true,% set by hyperref based on driver
   colorlinks=true,%
   allcolors=black,%
   pdfpagelayout=TwoPageRight,%
   pdfstartview=Fit%
}
\fi
%    \end{macrocode}
% enables correct jumping to figures when referencing
%    \begin{macrocode}
\RequirePackage[all]{hypcap}

\def\and{\unskip\,}

\iflnienglish
   \bibliographystyle{lni}
\else
   \bibliographystyle{lnig}
\fi

\frenchspacing    % Abstand nach Satzende
\widowpenalty=10000   % Vermeidet "widows & orphans"
\clubpenalty=10000
%    \end{macrocode}
%    \begin{macrocode}
%</class>
%    \end{macrocode}
%\Finale
%
%\iffalse
%
%    \begin{macrocode}
%<*bibtex>
%<*eng|ger>
%    \end{macrocode}
%    \begin{macrocode}
ENTRY
  { address
    author
    booktitle
    chapter
    edition
    editor
    howpublished
    institution
    journal
    key
    month
    note
    number
    organization
    pages
    publisher
    school
    series
    title
    type
    volume
    year
  }
  {}
  { label extra.label sort.label }

INTEGERS { output.state before.all mid.sentence after.sentence after.block }

FUNCTION {init.state.consts}
{ #0 'before.all :=
  #1 'mid.sentence :=
  #2 'after.sentence :=
  #3 'after.block :=
}

STRINGS { s t }

FUNCTION {output.nonnull}
{ 's :=
  output.state mid.sentence =
    { ", " * write$ }
    { output.state after.block =
	{ add.period$ write$
	  newline$
	  "\newblock " write$
	}
	{ output.state before.all =
	    'write$
	    { add.period$ " " * write$ }
	  if$
	}
      if$
      mid.sentence 'output.state :=
    }
  if$
  s
}

FUNCTION {output}
{ duplicate$ empty$
    'pop$
    'output.nonnull
  if$
}

FUNCTION {output.check}
{ 't :=
  duplicate$ empty$
    { pop$ "empty " t * " in " * cite$ * warning$ }
    'output.nonnull
  if$
}

FUNCTION {output.bibitem}
{ newline$
  "\bibitem[" write$
  label write$
  "]{" write$
  cite$ write$
  "}" write$
  newline$
  ""
  before.all 'output.state :=
}

FUNCTION {fin.entry}
{ add.period$
  write$
  newline$
}

FUNCTION {new.block}
{ output.state before.all =
    'skip$
    { after.block 'output.state := }
  if$
}

FUNCTION {new.sentence}
{ output.state after.block =
    'skip$
    { output.state before.all =
	'skip$
	{ after.sentence 'output.state := }
      if$
    }
  if$
}

FUNCTION {not}
{   { #0 }
    { #1 }
  if$
}

FUNCTION {and}
{   'skip$
    { pop$ #0 }
  if$
}

FUNCTION {or}
{   { pop$ #1 }
    'skip$
  if$
}

FUNCTION {new.block.checka}
{ empty$
    'skip$
    'new.block
  if$
}

FUNCTION {new.block.checkb}
{ empty$
  swap$ empty$
  and
    'skip$
    'new.block
  if$
}

FUNCTION {new.sentence.checka}
{ empty$
    'skip$
    'new.sentence
  if$
}

FUNCTION {new.sentence.checkb}
{ empty$
  swap$ empty$
  and
    'skip$
    'new.sentence
  if$
}

FUNCTION {field.or.null}
{ duplicate$ empty$
    { pop$ "" }
    'skip$
  if$
}

FUNCTION {emphasize}
{ duplicate$ empty$
    { pop$ "" }
    { "{\em " swap$ * "}" * }
  if$
}

INTEGERS { nameptr namesleft numnames }

FUNCTION {format.names} %formats the names of the authors
{ 's :=
  #1 'nameptr :=
  s num.names$ 'numnames :=
  numnames 'namesleft :=
    { namesleft #0 > }
    { s nameptr "{vv~}{ll}{, jj}{, ff}" format.name$ 't 
% last name, prename(s)
     :=
      nameptr #1 >
        { namesleft #1 >
            { "; " * t * }
%
% (RT) Now here we have discovered a bug. The bibliography of the LaTeX-book shows
% the fault: If there are three or more authors, then the final ``and'' before
% the last author's name is preceeded by a comma. This looks ugly and is
% wrong.
% The original and faulty line in format.names reads
%            { numnames #2 >
% Actually the generation of the comma is dependent on the number of names
% left. Therefore:
%
            { namesleft #2 >
                { ";" * }
                'skip$
              if$
              t "others" =
                { " et~al." * }
% ``and'' changed to ``;'' 
                { "; " * t * }
              if$
            }
          if$
        }
        't
      if$
      nameptr #1 + 'nameptr :=
      namesleft #1 - 'namesleft :=
    }
  while$
   
}

FUNCTION {format.authors}
{ author empty$
    { "" }
    { author format.names }
  if$
  ": " * %add a : after the authors
}

FUNCTION {format.editors}
{ editor empty$
    { "" }
    { editor format.names
      editor num.names$ #1 >
% use abbreviations
%</eng|ger>
%<*eng>
	{ ", eds" * }
	{ ", ed." * }
%</eng>
%<*ger>
	{ ", Hrsg." * }
	{ ", Hrsg." * }
%</ger>
%<*eng|ger>	
      if$
    }
  if$
}

FUNCTION {format.title}
{ title empty$
    { "" }
% (RT) the title as typed in the database here 
%    { title "t" change.case$ }
    'title
  if$
}

FUNCTION {n.dashify}
{ 't :=
  ""
    { t empty$ not }
    { t #1 #1 substring$ "-" =
	{ t #1 #2 substring$ "--" = not
	    { "--" *
	      t #2 global.max$ substring$ 't :=
	    }
	    {   { t #1 #1 substring$ "-" = }
		{ "-" *
		  t #2 global.max$ substring$ 't :=
		}
	      while$
	    }
	  if$
	}
	{ t #1 #1 substring$ *
	  t #2 global.max$ substring$ 't :=
	}
      if$
    }
  while$
}

FUNCTION {format.date}
{ year empty$
    { month empty$
	{ "" }
	{ "there's a month but no year in " cite$ * warning$
	  month
	}
      if$
    }
    { month empty$
	'year
	{ month " " * year * }
      if$
    }
  if$
}

FUNCTION {format.btitle}
{ title % no emphasize
}

FUNCTION {tie.or.space.connect}
{ duplicate$ text.length$ #3 <
    { "~" }
    { " " }
  if$
  swap$ * *
}

FUNCTION {either.or.check}
{ empty$
    'pop$
    { "can't use both " swap$ * " fields in " * cite$ * warning$ }
  if$
}

FUNCTION {format.bvolume}
{ volume empty$
    { "" }
%</eng|ger>
%<*eng>
    { "volume" volume tie.or.space.connect
%</eng>
%<*ger>    
    { "Jgg." volume tie.or.space.connect
%</ger>
%<*eng|ger>    
      series empty$
	'skip$
%</eng|ger>
%<*eng>	
	{ " of " * series % no emphasize 
%</eng>
%<*ger>
	{ " in " * series % no emphasize 
%</ger>
%<*eng|ger>	 	
	* }
      if$
      "volume and number" number either.or.check
    }
  if$
}

FUNCTION {format.number.series}
{ volume empty$
    { number empty$
	{ series field.or.null }
	{ output.state mid.sentence =
	    { "" } %No word before number
	    { "" }
	  if$
	  series empty$
        { "" }
        { series * }
      if$          
          number tie.or.space.connect % series and then the number, eg LNI 001
        }
    if$
    }
    { "" }
  if$
}

FUNCTION {format.edition}
{ edition empty$
    { "" }
    { output.state mid.sentence =
%</eng|ger>
%<*eng>    
	{ edition "l" change.case$ " edition" * }
	{ edition "t" change.case$ " edition" * }
%</eng>
%<*ger>
	{ edition "l" change.case$ ".~Auflage" * }
	{ edition "t" change.case$ ".~Auflage" * }
%</ger>
%<*eng|ger>	
      if$
    }
  if$
}

INTEGERS { multiresult }

FUNCTION {multi.page.check}
{ 't :=
  #0 'multiresult :=
    { multiresult not
      t empty$ not
      and
    }
    { t #1 #1 substring$
      duplicate$ "-" =
      swap$ duplicate$ "," =
      swap$ "+" =
      or or
	{ #1 'multiresult := }
	{ t #2 global.max$ substring$ 't := }
      if$
    }
  while$
  multiresult
}

FUNCTION {format.pages}
{ pages empty$
    { "" }
    { pages multi.page.check
%</eng|ger>
%<*eng>
    %  -> Use abbreviation: pp. for pages, p. for page
	{ "pp." pages n.dashify tie.or.space.connect }
	{ "p." pages tie.or.space.connect }
%</eng>
%<*ger>
    %  -> Nutze Abk.: S.
	{ "S." pages n.dashify tie.or.space.connect }
	{ "S." pages tie.or.space.connect }
%</ger>
%<*eng|ger>	
      if$
    }
  if$
}

FUNCTION {format.vol.num.pages}
{ volume field.or.null
  number empty$
    'skip$
    { "(" number * ")" * *
      volume empty$
	{ "there's a number but no volume in " cite$ * warning$ }
	'skip$
      if$
    }
  if$
  pages empty$
    'skip$
    { duplicate$ empty$
	{ pop$ format.pages }
	{ ":" * pages n.dashify * }
      if$
    }
  if$
}

FUNCTION {format.chapter.pages}
{ chapter empty$
    'format.pages
    { type empty$
%</eng|ger>
%<*eng>    
	{ "chapter" }
%</eng>
%<*ger>
	{ "Kapitel" }
%</ger>
%<*eng|ger>	
	{ type "l" change.case$ }
      if$
      chapter tie.or.space.connect
      pages empty$
	'skip$
	{ ", " * format.pages * }
      if$
    }
  if$
}

FUNCTION {format.in.ed.booktitle}
{ booktitle empty$
    { "" }
    { editor empty$
% no emphasize    
%       { "In: " booktitle emphasize * }
%       { "In: " format.editors * ", " * booktitle emphasize * }
        { "In: " booktitle  * }
        { "In (" format.editors * "): " * booktitle  * }
      if$
    }
  if$
}

FUNCTION {empty.misc.check}
{ author empty$ title empty$ howpublished empty$
  month empty$ year empty$ note empty$
  and and and and and
  key empty$ not and
    { "all relevant fields are empty in " cite$ * warning$ }
    'skip$
  if$
}

FUNCTION {format.thesis.type}
{ type empty$
    'skip$
    { pop$
      type "t" change.case$
    }
  if$
}

FUNCTION {format.tr.number}
{ type empty$
%</eng|ger>
%<*eng>
    { "Technical Report" }
%</eng>
%<*ger>
    { "Bericht" }
%</ger>
%<*eng|ger>    
    'type
  if$
  number empty$
    { "t" change.case$ }
    { number tie.or.space.connect }
  if$
}

FUNCTION {format.article.crossref}
{ key empty$
    { journal empty$
	{ "need key or journal for " cite$ * " to crossref " * crossref *
	  warning$
	  ""
	}
	{ "In {\em " journal * "\/}" * }
      if$
    }
    { "In " key * }
  if$
  " \cite{" * crossref * "}" *
}

FUNCTION {format.crossref.editor}
{ editor #1 "{vv~}{ll}" format.name$
  editor num.names$ duplicate$
  #2 >
    { pop$ " et~al." * }
    { #2 <
	'skip$
	% change order of names
	{ editor #2 "{vv~}{ll}{, jj}{, f.}" format.name$ "others" =
	    { " et~al." * }
	    % ; between two editors and change order of names
        { "; " * editor #2 "{vv~}{ll}{, jj}{, f.}" format.name$ * }
	  if$
	}
      if$
    }
  if$
}

FUNCTION {format.book.crossref}
{ volume empty$
    { "empty volume in " cite$ * "'s crossref of " * crossref * warning$
      "In "
    }
%</eng|ger>
%<*eng>
    { "Volume" volume tie.or.space.connect
      " of " *
    }
%</eng>
%<*ger>
    { "Jgg." volume tie.or.space.connect
      " in " *
    }
%</ger>
%<*eng|ger>    
  if$
  editor empty$
  editor field.or.null author field.or.null =
  or
    { key empty$
	{ series empty$
	    { "need editor, key, or series for " cite$ * " to crossref " *
	      crossref * warning$
	      "" *
	    }
	    { "{\em " * series * "\/}" * }
	  if$
	}
	{ key * }
      if$
    }
    { format.crossref.editor * }
  if$
  " \cite{" * crossref * "}" *
}

FUNCTION {format.incoll.inproc.crossref}
{ editor empty$
  editor field.or.null author field.or.null =
  or
    { key empty$
	{ booktitle empty$
	    { "need editor, key, or booktitle for " cite$ * " to crossref " *
	      crossref * warning$
	      ""
	    }
	    % Add : after In
	    { "In: {\em " booktitle * "\/}" * }
	  if$
	}
	% Add : after In
	{ "In: " key * }
      if$
    }
    % Add : after In and editiors in ()
    { "In: (" format.crossref.editor * ")" * }
  if$
  " \cite{" * crossref * "}" *
}

FUNCTION {article}
{ output.bibitem
  format.authors %"author" output.check %% No dot after authors
  new.block
  format.title "title" output.check
  new.block
  crossref missing$
    { journal % no emphasize 
    "journal" output.check
      format.vol.num.pages output
      format.date "year" output.check
    }
    { format.article.crossref output.nonnull
      format.pages output
    }
  if$
  new.block
  note output
  fin.entry
}

FUNCTION {book}
{ output.bibitem
  author empty$
    { format.editors "author and editor" output.check }
    { format.authors %output.nonnull %%no dot after authors:
      crossref missing$
	{ "author and editor" editor either.or.check }
	'skip$
      if$
    }
  if$
  new.block
  format.btitle "title" output.check
  crossref missing$
    { format.bvolume output
      new.block
      format.number.series output
      new.sentence
      publisher "publisher" output.check
      address output
    }
    { new.block
      format.book.crossref output.nonnull
    }
  if$
  format.edition output
  format.date "year" output.check
  new.block
  note output
  fin.entry
}

FUNCTION {booklet}
{ output.bibitem
  format.authors output
  new.block
  format.title "title" output.check
  howpublished address new.block.checkb
  howpublished output
  address output
  format.date output
  new.block
  note output
  fin.entry
}

FUNCTION {inbook}
{ output.bibitem
  author empty$
    { format.editors "author and editor" output.check }
    { format.authors %*output.nonnull %%no dot after authors:
      crossref missing$
	{ "author and editor" editor either.or.check }
	'skip$
      if$
    }
  if$
  new.block
  format.btitle "title" output.check
  new.block
  crossref missing$ %changed whole order of next lines
    { 
      format.in.ed.booktitle "booktitle" output.check %added: mention title of the related book and editors
      format.bvolume output
      new.block
      format.number.series output
      % new.sentence %% no dot after the number
      publisher "publisher" output.check
      address output
    }
    { format.chapter.pages "chapter and pages" output.check
      new.block
      format.book.crossref output.nonnull
    }
  if$
  format.chapter.pages "chapter and pages" output.check
  format.edition output
  format.date "year" output.check
  new.block
  note output
  fin.entry
}

FUNCTION {incollection}
{ output.bibitem
  format.authors %"author" output.check %%no dot after authors:
  new.block
  format.title "title" output.check
  new.block
  crossref missing$
    { format.in.ed.booktitle "booktitle" output.check
      format.bvolume output
      format.number.series output
      format.chapter.pages output
      new.sentence
      publisher "publisher" output.check
      address output
      format.edition output
      format.date "year" output.check
    }
    { format.incoll.inproc.crossref output.nonnull
      format.chapter.pages output
    }
  if$
  new.block
  note output
  fin.entry
}

FUNCTION {inproceedings}
{ output.bibitem
  format.authors %"author" output.check %%no dot after authors:
  new.block
  format.title "title" output.check
  new.block
  crossref missing$
    { format.in.ed.booktitle "booktitle" output.check
      new.block      
      format.bvolume output
      format.number.series output
      address empty$
	{ organization publisher new.sentence.checkb
	  organization output
	  publisher output
	}
	{ organization output
          publisher output
          address output.nonnull
        }
      if$
      format.pages output
      format.date "year" output.check
    }
    { format.incoll.inproc.crossref output.nonnull
      format.pages output
    }
  if$
  new.block
  note output
  fin.entry
}

FUNCTION {conference} { inproceedings }

FUNCTION {manual}
{ output.bibitem
  author empty$
    { organization empty$
	'skip$
	{ organization output.nonnull
	  address output
	}
      if$
    }
    { format.authors output.nonnull }
  if$
  new.block
  format.btitle "title" output.check
  author empty$
    { organization empty$
	{ address new.block.checka
	  address output
	}
	'skip$
      if$
    }
    { organization address new.block.checkb
      organization output
      address output
    }
  if$
  format.edition output
  format.date output
  new.block
  note output
  fin.entry
}

FUNCTION {mastersthesis}
{ output.bibitem
  format.authors %"author" output.check
  new.block
  format.title "title" output.check
  new.block
%</eng|ger>
%<*eng>
  "Master's thesis" format.thesis.type output.nonnull
%</eng>
%<*ger>
  "Masterarbeit" format.thesis.type output.nonnull
%</ger>
%<*eng|ger>  
  school "school" output.check
  address output
  format.date "year" output.check
  new.block
  note output
  fin.entry
}

FUNCTION {misc}
{ output.bibitem
  author empty$
    { ""  }
    { format.authors output.nonnull }
  if$
  %title howpublished new.block.checkb
  format.title output
  howpublished new.block.checka
  howpublished output
  format.date output
  new.block
  note output
  fin.entry
  empty.misc.check
}

FUNCTION {online}
{ output.bibitem
  title howpublished new.block.checkb
  format.title output
  howpublished new.block.checka
  howpublished output
  note output
  fin.entry
  empty.misc.check
}

FUNCTION {phdthesis}
{ output.bibitem
  format.authors %"author" output.check
  new.block
  format.btitle "title" output.check
  new.block
%</eng|ger>
%<*eng>
  "PhD thesis" format.thesis.type output.nonnull
%</eng>
%<*ger>
  "Dissertation" format.thesis.type output.nonnull
%</ger>
%<*eng|ger>  
  school "school" output.check
  address output
  format.date "year" output.check
  new.block
  note output
  fin.entry
}

FUNCTION {proceedings}
{ output.bibitem
  editor empty$
    { organization output }
    { format.editors output.nonnull }
  if$
  new.block
  format.btitle "title" output.check
  format.bvolume output
  format.number.series output
  address empty$
    { editor empty$
	{ publisher new.sentence.checka }
	{ organization publisher new.sentence.checkb
	  organization output
	}
      if$
      publisher output
      format.date "year" output.check
    }
    { address output.nonnull
      format.date "year" output.check
      new.sentence
      editor empty$
	'skip$
	{ organization output }
      if$
      publisher output
    }
  if$
  new.block
  note output
  fin.entry
}

FUNCTION {techreport}
{ output.bibitem
  format.authors %"author" output.check
  new.block
  format.title "title" output.check
  new.block
  format.tr.number output.nonnull
  institution "institution" output.check
  address output
  format.date "year" output.check
  new.block
  note output
  fin.entry
}

FUNCTION {unpublished}
{ output.bibitem
  format.authors %"author" output.check
  new.block
  format.title "title" output.check
  new.block
  note "note" output.check
  format.date output
  fin.entry
}

FUNCTION {default.type} { misc }

%</eng|ger>
%<*eng>
MACRO {jan} {"January"}

MACRO {feb} {"February"}

MACRO {mar} {"March"}

MACRO {apr} {"April"}

MACRO {may} {"May"}

MACRO {jun} {"June"}

MACRO {jul} {"July"}

MACRO {aug} {"August"}

MACRO {sep} {"September"}

MACRO {oct} {"October"}

MACRO {nov} {"November"}

MACRO {dec} {"December"}
%</eng>
%<*ger>
MACRO {jan} {"Januar"}

MACRO {feb} {"Februar"}

MACRO {mar} {"März"}

MACRO {apr} {"April"}

MACRO {may} {"Mai"}

MACRO {jun} {"Juni"}

MACRO {jul} {"Juli"}

MACRO {aug} {"August"}

MACRO {sep} {"September"}

MACRO {oct} {"Oktober"}

MACRO {nov} {"November"}

MACRO {dec} {"Dezember"}
%</ger>
%<*eng|ger>

MACRO {acmcs} {"ACM Computing Surveys"}

MACRO {acta} {"Acta Informatica"}

MACRO {cacm} {"Communications of the ACM"}

MACRO {ibmjrd} {"IBM Journal of Research and Development"}

MACRO {ibmsj} {"IBM Systems Journal"}

MACRO {ieeese} {"IEEE Transactions on Software Engineering"}

MACRO {ieeetc} {"IEEE Transactions on Computers"}

MACRO {ieeetcad}
 {"IEEE Transactions on Computer-Aided Design of Integrated Circuits"}

MACRO {ipl} {"Information Processing Letters"}

MACRO {jacm} {"Journal of the ACM"}

MACRO {jcss} {"Journal of Computer and System Sciences"}

MACRO {scp} {"Science of Computer Programming"}

MACRO {sicomp} {"SIAM Journal on Computing"}

MACRO {tocs} {"ACM Transactions on Computer Systems"}

MACRO {tods} {"ACM Transactions on Database Systems"}

MACRO {tog} {"ACM Transactions on Graphics"}

MACRO {toms} {"ACM Transactions on Mathematical Software"}

MACRO {toois} {"ACM Transactions on Office Information Systems"}

MACRO {toplas} {"ACM Transactions on Programming Languages and Systems"}

MACRO {tcs} {"Theoretical Computer Science"}

READ

FUNCTION {sortify}
{ purify$
  "l" change.case$
}

INTEGERS { len }

FUNCTION {chop.word}
{ 's :=
  'len :=
  s #1 len substring$ =
    { s len #1 + global.max$ substring$ }
    's
  if$
}

INTEGERS { et.al.char.used }

FUNCTION {initialize.et.al.char.used}
{ #0 'et.al.char.used :=
}

EXECUTE {initialize.et.al.char.used}

FUNCTION {format.lab.names}
{ 's :=
  s num.names$ 'numnames :=
  numnames #1 >
    { numnames #3 > %create short entry for 3 authors max
        { #1 'namesleft := } %if more than 3 authors take only first one
        { numnames 'namesleft := }
      if$
      #1 'nameptr :=
      ""
        { namesleft #0 > }
        { nameptr numnames =
            { s nameptr "{ff }{vv }{ll}{ jj}" format.name$ 
            "others" =
                { s #1 "{ll}" format.name$ #2 text.prefix$ %% second sign of first author if there are others
                }
                { s nameptr "{v{}}{l{}}" format.name$ * } %%formats last author
              if$
            }
            { s nameptr "{v{}}{l{}}" format.name$ * } %%formats first, and if three, also the second author
          if$
          nameptr #1 + 'nameptr :=
          namesleft #1 - 'namesleft :=
        }
      while$
      numnames #3 > 
        { s #1 "{ll}" format.name$ #2 text.prefix$ } % if more than 3 authors take second letter of first name
        'skip$
      if$
    }
    { s #1 "{v{}}{l{}}" format.name$
      duplicate$ text.length$ #2 <
% Das dritte Zeichen?      
        { pop$ s #1 "{ll}" format.name$ #2 text.prefix$} %get rid of the rest of the name (only first two letters are needed for one author)
        'skip$
      if$
    }
  if$
}

FUNCTION {author.key.label} %% sets the author abbreviation in []
{ author empty$
    { key empty$
	{ title #1 #2 substring$ } % Use First two letters of title if there is no author
	{ key #3 text.prefix$ }
      if$
    }
    { author format.lab.names }
  if$
}

FUNCTION {author.editor.key.label}
{ author empty$
    { editor empty$
	{ key empty$
	    { title #1 #2 substring$ }% Use First two letters of title if there is no editor
	    { key #3 text.prefix$ }
	  if$
	}
	{ editor format.lab.names }
      if$
    }
    { author format.lab.names }
  if$
}

FUNCTION {author.key.organization.label}
{ author empty$
    { key empty$
	{ organization empty$
	    { cite$ #1 #3 substring$ }
	    { "The " #4 organization chop.word #3 text.prefix$ }
	  if$
	}
	{ key #3 text.prefix$ }
      if$
    }
    { author format.lab.names }
  if$
}

FUNCTION {editor.key.organization.label}
{ editor empty$
    { key empty$
	{ organization empty$
	    { cite$ #1 #3 substring$ }
	    { "The " #4 organization chop.word #3 text.prefix$ }
	  if$
	}
	{ key #3 text.prefix$ }
      if$
    }
    { editor format.lab.names }
  if$
}

FUNCTION {calc.label}
{ type$ "book" =
  type$ "inbook" =
  or
    'author.editor.key.label
    { type$ "proceedings" =
	'editor.key.organization.label
	{ type$ "manual" =
	    'author.key.organization.label
	    'author.key.label
	  if$
	}
      if$
    }
  if$
  duplicate$
  year field.or.null purify$ #-1 #2 substring$
  *
  'label :=
  year field.or.null purify$ #-1 #4 substring$
  *
  sortify 'sort.label :=
}

FUNCTION {sort.format.names}
{ 's :=
  #1 'nameptr :=
  ""
  s num.names$ 'numnames :=
  numnames 'namesleft :=
    { namesleft #0 > }
    { nameptr #1 >
	{ "   " * }
	'skip$
      if$
      s nameptr "{vv{ } }{ll{ }}{  ff{ }}{  jj{ }}" format.name$ 't :=
      nameptr numnames = t "others" = and
	{ "et al" * }
	{ t sortify * }
      if$
      nameptr #1 + 'nameptr :=
      namesleft #1 - 'namesleft :=
    }
  while$
}

FUNCTION {sort.format.title}
{ 't :=
  "A " #2
    "An " #3
      "The " #4 t chop.word
    chop.word
  chop.word
  sortify
  #1 global.max$ substring$
}

FUNCTION {author.sort}
{ author empty$
    { key empty$
	{ "to sort, need author or key in " cite$ * warning$
	  ""
	}
	{ key sortify }
      if$
    }
    { author sort.format.names }
  if$
}

FUNCTION {author.editor.sort}
{ author empty$
    { editor empty$
	{ key empty$
	    { "to sort, need author, editor, or key in " cite$ * warning$
	      ""
	    }
	    { key sortify }
	  if$
	}
	{ editor sort.format.names }
      if$
    }
    { author sort.format.names }
  if$
}

FUNCTION {author.organization.sort}
{ author empty$
    { organization empty$
	{ key empty$
	    { "to sort, need author, organization, or key in " cite$ * warning$
	      ""
	    }
	    { key sortify }
	  if$
	}
	{ "The " #4 organization chop.word sortify }
      if$
    }
    { author sort.format.names }
  if$
}

FUNCTION {editor.organization.sort}
{ editor empty$
    { organization empty$
	{ key empty$
	    { "to sort, need editor, organization, or key in " cite$ * warning$
	      ""
	    }
	    { key sortify }
	  if$
	}
	{ "The " #4 organization chop.word sortify }
      if$
    }
    { editor sort.format.names }
  if$
}

FUNCTION {presort}
{ calc.label
  sort.label
  "    "
  *
  type$ "book" =
  type$ "inbook" =
  or
    'author.editor.sort
    { type$ "proceedings" =
	'editor.organization.sort
	{ type$ "manual" =
	    'author.organization.sort
	    'author.sort
	  if$
	}
      if$
    }
  if$
  *
  "    "
  *
  year field.or.null sortify
  *
  "    "
  *
  title field.or.null
  sort.format.title
  *
  #1 entry.max$ substring$
  'sort.key$ :=
}

ITERATE {presort}

SORT

STRINGS { longest.label last.sort.label next.extra }

INTEGERS { longest.label.width last.extra.num }

FUNCTION {initialize.longest.label}
{ "" 'longest.label :=
  #0 int.to.chr$ 'last.sort.label :=
  "" 'next.extra :=
  #0 'longest.label.width :=
  #0 'last.extra.num :=
}

FUNCTION {forward.pass}
{ last.sort.label sort.label =
    { last.extra.num #1 + 'last.extra.num :=
      last.extra.num int.to.chr$ 'extra.label :=
    }
    { "a" chr.to.int$ 'last.extra.num :=
      "" 'extra.label :=
      sort.label 'last.sort.label :=
    }
  if$
}

FUNCTION {reverse.pass}
{ next.extra "b" =
    { "a" 'extra.label := }
    'skip$
  if$
  label extra.label * 'label :=
  label width$ longest.label.width >
    { label 'longest.label :=
      label width$ 'longest.label.width :=
    }
    'skip$
  if$
  extra.label 'next.extra :=
}

EXECUTE {initialize.longest.label}

ITERATE {forward.pass}

REVERSE {reverse.pass}

FUNCTION {begin.bib}
{ et.al.char.used
% to make it compatible with singlepass, we use \def
%    { "\newcommand{\etalchar}[1]{$^{#1}$}" write$ newline$ }
    { "\gdef\etalchar#1{$^{#1}$}" write$ newline$ } 
    'skip$
  if$
  preamble$ empty$
    'skip$
    { preamble$ write$ newline$ }
  if$
  "\begin{thebibliography}{"  longest.label  * "}" * write$ newline$
}

EXECUTE {begin.bib}

EXECUTE {init.state.consts}

ITERATE {call.type$}

FUNCTION {end.bib}
{ newline$
  "\end{thebibliography}" write$ newline$
}

EXECUTE {end.bib}
%</eng|ger>
%</bibtex>
%<*template>
%% !TeX encoding = UTF-8
%% !TeX program = pdflatex
%% !BIB program = bibtex
%%
%%% Um einen Artikel auf deutsch zu schreiben, genügt es die Klasse ohne 
%%% Parameter zu laden.
\documentclass[]{lni}
%%% To write an article in English, please use the option ``english'' in order 
%%% to get the correct hyphenation patterns and terms.
%%% \documentclass[english]{class}
%%
\begin{document}
%%% Mehrere Autoren werden durch \and voneinander getrennt.
%%% Die Fußnote enthält die Adresse sowie eine E-Mail-Adresse.
%%% Das optionale Argument (sofern angegeben) wird für die Kopfzeile verwendet.
\author[Vorname1 Name1\and Vorname2 Name 2]{Vorname1 
Nachname1\footnote{Einrichtung/Universität, Abteilung, Anschrift, Postleitzahl 
Ort, \email{emailadresse@author1}}\and Vorname2 
Nachname2\footnote{Einrichtung/Universität, Abteilung, Anschrift, Postleitzahl 
Ort, \email{emailadresse@author2}} und weitere Autorinnen und Autoren in der 
gleichen Notation}
%
\title[Kurztitel (falls nötig)]{Titel}
%
\startpage{11}% Beginn der Seitenzählung für diesen Beitrag
\editor{Ich selbst}
\booktitle{Ein Test}
\year{2017}
\maketitle

\begin{abstract}
Hier kommt die Zusammenfassung hin
Add your abstract here
\end{abstract}
\begin{keywords}
Hier kommen die Keywords hin
Add your keywords here
\end{keywords}
%%% Beginn des Artikeltexts
\section{Überschrift}

%%% Angabe der .bib-Datei (ohne Endung)
\bibliography{mybibfile}
\end{document}
%</template>
%    \end{macrocode}
%\fi 