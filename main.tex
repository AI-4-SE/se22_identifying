% !TeX encoding = UTF-8
% !TeX spellcheck = de_DE

%% Dies gibt Warnungen aus, sollten veraltete LaTeX-Befehle verwendet werden
\RequirePackage[l2tabu, orthodox]{nag}

\documentclass[utf8,biblatex]{lni}
\bibliography{lni-paper-example-de}

%% Schöne Tabellen mittels \toprule, \midrule, \bottomrule
\usepackage{booktabs}


%% Zu Demonstrationszwecken
\usepackage[math]{blindtext}
\usepackage{mwe}

%\usepackage[subtle]{savetrees}
\usepackage{microtype}
\usepackage{paralist}
\usepackage{wrapfig}

\usepackage{blindtext}


%% BibLaTeX-Sonderkonfiguration,
%% falls man schnell eine existierende Bibliographie wiederverwenden will, aber nicht die .bib-Datei händisch anpassen möchte.
%% Bitte \iffalse und \fi entfernen, dann ist diese Konfiguration aktiviert.

\iffalse
\AtEveryBibitem{%
  \ifentrytype{article}{%
  }{%
    \clearfield{doi}%
    \clearfield{issn}%
    \clearfield{url}%
    \clearfield{urldate}%
  }%
  \ifentrytype{inproceedings}{%
  }{%
    \clearfield{doi}%
    \clearfield{issn}%
    \clearfield{url}%
    \clearfield{urldate}%
  }%
}
\fi

\begin{document}
%%% Mehrere Autoren werden durch \and voneinander getrennt.
%%% Die Fußnote enthält die Adresse sowie eine E-Mail-Adresse.
%%% Das optionale Argument (sofern angegeben) wird für die Kopfzeile verwendet.
\title[Identifying Software Performance Changes Across Variants and Versions]{Identifying Software Performance Changes\\ Across Variants and Versions}
%%%\subtitle{Untertitel / Subtitle} % falls benötigt
\author[Stefan Mühlbauer \and Sven Apel \and Norbert Siegmund]
{Stefan Mühlbauer\footnote{Universität Leipzig, Institut für Informatik, Leipzig, Deutschland, \email{muehlbauer@informatik.uni-leipzig.de}} \and
Sven Apel\textbf{\footnote{Universität des Saarlandes, Saarland Informatics Campus, Saarbrücken, Deutschland, \email{apel@cs.uni-saarland.de}}} \and
Norbert Siegmund\footnote{Universität Leipzig, Institut für Informatik, Leipzig, Deutschland, \email{norbert.siegmund@informatik.uni-leipzig.de}}
}
\startpage{11} % Beginn der Seitenzählung für diesen Beitrag
\editor{Herausgeber et al.}    % Namen der Herausgeber
\booktitle{Name-der-Konferenz} % Name des Taguwrapfigngsband; optional Kurztitel
\yearofpublication{2017}
%%%\lnidoi{18.18420/provided-by-editor-02} % Falls bekannt
\maketitle

\begin{abstract}
%{\color{blue!75!black}The original paper was published under the same title at ASE'20~\cite{muehlbauer_identifying_2020}.}
Performance changes of configurable software systems can occur and persist throughout its lifetime. Finding optimal configurations and configuration options that influence performance is already difficult, but in the light of software evolution, configuration-dependent performance changes may lurk in a potentially large number of different versions of the system. 
Building on previous work, we combine two perspectives---variability and time---and devise an approach to identify configuration-dependent performance changes retrospectively across the software variants and versions of a software system. In a nutshell, we iteratively sample pairs of configurations and versions and measure the respective performance, which we use to actively learn a model to estimate how likely a commit introduces a performance change. For such commits, we infer the configuration options that best explain observed performance changes. 
Pursuing a search strategy with the goal of measuring selectively and incrementally further pairs, we increase the accuracy of identified change points related to configuration options and interactions.
Our evaluation with both real-world software systems and synthesized data demonstrates that we can pinpoint performance shifts to individual configuration options and commits with high accuracy and at scale. 
\end{abstract}

\begin{keywords}
Software Performance; Configurable Software Systems; Software Evolution
\end{keywords}

%\textbf{\color{purple}Motivation/Problem} 
Modern software systems often provide configuration options to customize functionality. Configuration decisions, however, can also influence the performance behavior of a system. Configuration-dependent performance behavior can be inferred from a subset of observations using machine learning, but performance influences can change along with software evolution. If such shifts remain undetected and persist, models from older versions make inaccurate estimations for newer versions. Maintaining accurate prediction models throughout a software's lifetime, for instance via transfer learning, incurs additional cost of measurement. By contrast, information about which options' influences change at what revision could guide practicioners to transfer or re-learn models only upon such an occurence. 
We address this problem and devise a novel approach that retrospectively identifies shifts in the performance influence of options and combinations among them across a software systems’s development history~\cite{muehlbauer_identifying_2020}. 

In previous work, we outline how one can identify performance shifts in a series of performance measurements across revisions of a single software configuration with Gaussian Processes~\cite{muhlbauer_accurate_2019}. Configuration-specific change points can emerge from change in the performance influence of one or more configuration options. Hence, to not only identify performance shifts in arbitrary configurations, we extend our perspective of previous work~\cite{muhlbauer_accurate_2019} and incorporate  configurability into our analysis. 
Starting from a small sample of performance measurements across random configurations, our active learning and sampling strategy \textit{first} estimates the temporal location of likely change points (candidate commits). If a performance shift is observed at a candidate commit for multiple configurations, the shift is marked for further analysis. \textit{Second}, we attribute each candidate shift to one or more configuration options based on for which configurations it manifests. \textit{Last}, to increase the reliability of these estimations, we iteratively feed new measurements to the training set and repeat the previous steps until we reach a steady state.
	
Akin to learning configuration-dependent performance models, the main challenge with this approach is to wisely select configurations and revisions for measurement. To tame the combinatorial complexity of the configuration space, a modestly sized sample set 


\begin{compactitem}
	\item how to spend the budge (\textit{exploration} vs \textit{exploitation})?
	\item explanation of active sampling and illustration of graphic
\end{compactitem}

\begin{wrapfigure}{R}{0.6\linewidth}
	\centering
	\includegraphics[width=\linewidth]{figs/sampling_plan}
\end{wrapfigure}

In a study with both real-world and synthesized data...

\vfill
\printbibliography

\end{document}
