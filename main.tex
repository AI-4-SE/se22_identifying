% !TeX encoding = UTF-8
% !TeX spellcheck = de_DE

%% Dies gibt Warnungen aus, sollten veraltete LaTeX-Befehle verwendet werden
\RequirePackage[l2tabu, orthodox]{nag}

\documentclass[utf8,biblatex]{lni}
\bibliography{lni-paper-example-de}

%% Schöne Tabellen mittels \toprule, \midrule, \bottomrule
\usepackage{booktabs}


%% Zu Demonstrationszwecken
\usepackage[math]{blindtext}
\usepackage{mwe}
\usepackage[subtle]{savetrees}
\usepackage{microtype}
\usepackage{paralist}
\usepackage{wrapfig}

\usepackage{blindtext}


%% BibLaTeX-Sonderkonfiguration,
%% falls man schnell eine existierende Bibliographie wiederverwenden will, aber nicht die .bib-Datei händisch anpassen möchte.
%% Bitte \iffalse und \fi entfernen, dann ist diese Konfiguration aktiviert.

\iffalse
\AtEveryBibitem{%
  \ifentrytype{article}{%
  }{%
    \clearfield{doi}%
    \clearfield{issn}%
    \clearfield{url}%
    \clearfield{urldate}%
  }%
  \ifentrytype{inproceedings}{%
  }{%
    \clearfield{doi}%
    \clearfield{issn}%
    \clearfield{url}%
    \clearfield{urldate}%
  }%
}
\fi

\begin{document}
%%% Mehrere Autoren werden durch \and voneinander getrennt.
%%% Die Fußnote enthält die Adresse sowie eine E-Mail-Adresse.
%%% Das optionale Argument (sofern angegeben) wird für die Kopfzeile verwendet.
\title[Identifying Software Performance Changes Across Variants and Versions]{Identifying Software Performance Changes\\ Across Variants and Versions}
%%%\subtitle{Untertitel / Subtitle} % falls benötigt
\author[Stefan Mühlbauer \and Sven Apel \and Norbert Siegmund]
{Stefan Mühlbauer\footnote{Universität Leipzig, Institut für Informatik, Leipzig, Deutschland, \email{muehlbauer@informatik.uni-leipzig.de}} \and
Sven Apel\textbf{\footnote{Universität des Saarlandes, Saarland Informatics Campus, Saarbrücken, Deutschland, \email{apel@cs.uni-saarland.de}}} \and
Norbert Siegmund\footnote{Universität Leipzig, Institut für Informatik, Leipzig, Deutschland, \email{norbert.siegmund@informatik.uni-leipzig.de}}
}
\startpage{11} % Beginn der Seitenzählung für diesen Beitrag
\editor{Herausgeber et al.}    % Namen der Herausgeber
\booktitle{Name-der-Konferenz} % Name des Taguwrapfigngsband; optional Kurztitel
\yearofpublication{2017}
%%%\lnidoi{18.18420/provided-by-editor-02} % Falls bekannt
\maketitle

\begin{abstract}
%{\color{blue!75!black}The original paper was published under the same title at ASE'20~\cite{muehlbauer_identifying_2020}.}
Performance changes of configurable software systems can occur and persist throughout their lifetime. Finding optimal configurations and configuration options that influence performance is already difficult, but in the light of software evolution, configuration-dependent performance changes may lurk in a potentially large number of different versions of the system. 
Building on previous work, we combine two perspectives---variability and time---and devise an approach to identify configuration-dependent performance changes retrospectively across the software variants and versions of a software system. In a nutshell, we iteratively sample pairs of configurations and versions and measure the respective performance, which we use to actively learn a model that estimates how likely a commit introduces a performance change. For such commits, we infer the configuration options that best explain observed performance changes. 
Pursuing a search strategy with the goal of measuring selectively and incrementally further pairs, we increase the accuracy of identified change points related to configuration options and interactions.
Our evaluation with both real-world software systems and synthesized data demonstrates that we can pinpoint performance shifts to individual configuration options and commits with high accuracy and at scale. 
\end{abstract}

\begin{keywords}
Software Performance; Configurable Software Systems; Software Evolution
\end{keywords}

%\textbf{\color{purple}Motivation/Problem} 
Modern software systems often provide configuration options to customize their functionality and non-functional characteristics, such as energy consumption and performance. 
Determining the influence of individual configuartion options and their interactions on performance is usually done in a two step-process. First, a set of configurations is selected and then measured via a benchmark or an application-specific workload. Second, a machine-learning technique, such as linear regression or classification and regression trees, are applied to learn a performance model using the measurements as training set. With the resulting model, we can estimate performance of unseen configurations or compute the performance optimal one.

However, software is not a static entity and as the code base changes so does performance. If the influence of options or interactions on performance change, prediction models from older versions make inaccurate estimations for newer versions. Maintaining accurate prediction models throughout a software's lifetime vastly increases the cost of measurement since we would blindly select configurations throughout the version history of a system to find the relevant change points.

By contrast, information about which options' influences change at what revision could guide practicioners to re-learn models only upon such an occurence or guide future testing efforts. 
We address the problem of detecting change points in space (configuration dimension) and time (version dimension) and devise a novel approach that \textit{retrospectively} identifies shifts in the performance influence of options and combinations among them across a software systems’s development history~\cite{muehlbauer_identifying_2020}. 

In previous work, we presented how one can identify performance shifts in a series of performance measurements of a single software configuration~\cite{muhlbauer_accurate_2019}.
Since configuration-specific performance changes often emerge from changes in the performance influence of one or more configuration options, it is key to identify such causative options. To this end, we extend our perspective from previous work~\cite{muhlbauer_accurate_2019} and pinpoint performance change not only to a specific commit, but also a set of responsible configuration options. 

Starting from performance measurements of a small sample of random configurations and revisions, our active learning and sampling strategy first estimates the temporal location of likely change points (candidate commits). If a performance shift is observed at a candidate commit for multiple configurations, the shift is marked for further analysis (candidate shift). Second, we attribute each candidate shift to one or more configuration options based on the configurations for which it manifests. Last, to increase the reliability of these estimations, we iteratively add new measurements to the training set and repeat the previous steps until the set of identified change points does not change over a couple of iterations.

The main challenge with this approach is the combinatorial complexity of the combined configuration-and-revision space. To wisely select configurations and revisions, we balance the budget of measurements between two objectives, \textit{exploration} and \textit{exploitation}. To be able to capture undetected performance shifts at all, we sample most configurations and revisions randomly ({exploration}), whereas we dedicate the remaining budget is to re-evaluating and refining the candidate change points that our approach yields in each iteration ({exploitation}). While the initial sample set is entirely random, we gradually increase the number of exploitation measurements with every iteration. 

%The cycle of iterations stops once the set of candidate solutions does not change over a number of iterations. 

In a study with three real-world software systems, we show that our approach can precisely pinpoint configuration-specific performance changes to a single commit and associate it with related configuration options. Further experiments with synthetic data demonstrate that our approach can operate at scale with regard to the number of commits of a version history, the number of configuration options, and the degree of option interactions.

\textbf{Data Availability}~~
We make all measurement data and a reference implementation of our approach~\cite{muehlbauer_identifying_2020} available in the original paper's replication package~\footnote{\url{https://archive.softwareheritage.org/browse/origin/directory/?origin_url=https://github.com/AI-4-SE/Changepoints-Across-Variants-And-Versions}}.
\printbibliography

\end{document}
