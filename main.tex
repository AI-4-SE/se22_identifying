% !TeX encoding = UTF-8
% !TeX spellcheck = de_DE

%% Dies gibt Warnungen aus, sollten veraltete LaTeX-Befehle verwendet werden
\RequirePackage[l2tabu, orthodox]{nag}

\documentclass[utf8,biblatex]{lni}
\bibliography{lni-paper-example-de}

%% Schöne Tabellen mittels \toprule, \midrule, \bottomrule
\usepackage{booktabs}

%% Zu Demonstrationszwecken
\usepackage[math]{blindtext}
\usepackage{mwe}

%\usepackage[subtle]{savetrees}
\usepackage{microtype}

%% BibLaTeX-Sonderkonfiguration,
%% falls man schnell eine existierende Bibliographie wiederverwenden will, aber nicht die .bib-Datei händisch anpassen möchte.
%% Bitte \iffalse und \fi entfernen, dann ist diese Konfiguration aktiviert.

\iffalse
\AtEveryBibitem{%
  \ifentrytype{article}{%
  }{%
    \clearfield{doi}%
    \clearfield{issn}%
    \clearfield{url}%
    \clearfield{urldate}%
  }%
  \ifentrytype{inproceedings}{%
  }{%
    \clearfield{doi}%
    \clearfield{issn}%
    \clearfield{url}%
    \clearfield{urldate}%
  }%
}
\fi

\begin{document}
%%% Mehrere Autoren werden durch \and voneinander getrennt.
%%% Die Fußnote enthält die Adresse sowie eine E-Mail-Adresse.
%%% Das optionale Argument (sofern angegeben) wird für die Kopfzeile verwendet.
\title[Identifying Software Performance Changes Across Variants and Versions]{Identifying Software Performance Changes\\ Across Variants and Versions}
%%%\subtitle{Untertitel / Subtitle} % falls benötigt
\author[Stefan Mühlbauer \and Sven Apel \and Norbert Siegmund]
{Stefan Mühlbauer\footnote{Universität Leipzig, Institut für Informatik, Leipzig, Deutschland, \email{muehlbauer@informatik.uni-leipzig.de}} \and
Sven Apel\textbf{\footnote{Universität des Saarlandes, Saarland Informatics Campus, Saarbrücken, Deutschland, \email{apel@cs.uni-saarland.de}}} \and
Norbert Siegmund\footnote{Universität Leipzig, Institut für Informatik, Leipzig, Deutschland, \email{norbert.siegmund@uni-leipzig.de}}
}
\startpage{11} % Beginn der Seitenzählung für diesen Beitrag
\editor{Herausgeber et al.}    % Namen der Herausgeber
\booktitle{Name-der-Konferenz} % Name des Tagungsband; optional Kurztitel
\yearofpublication{2017}
%%%\lnidoi{18.18420/provided-by-editor-02} % Falls bekannt
\maketitle

\begin{abstract}
The original paper was published under the title~\citetitle{muehlbauer_identifying_2020} in the proceedings of the ASE'20.

TBD~\cite{muhlbauer_accurate_2019,muehlbauer_identifying_2020}
\end{abstract}

\begin{keywords}
Software Performance; Configurable Software Systems; Software Evolution
\end{keywords}



%% \bibliography{lni-paper-example-de.tex} ist hier nicht erlaubt: biblatex erwartet dies bei der Preambel
%% Starten Sie "biber paper", um eine Biliographie zu erzeugen.
\printbibliography

\end{document}
